\subsection{Other Annotations}

\subsubsection{Correcting for Attrition}
\noindent It is very common for economists to correct for attrition. This is, to correct for the fact that some variables are not observed for certain individuals. A usual way to do it is to estimate a model predicting the probability of attrition based on observed characteristics. For example, if income is an outcome of interest and the researcher does not observe income for a subset of the sample but has observed characteristics for the complete sample, she can predict the probability of attrition. Then, she can use a method such as inverse probability weighting (IPW) to give greater relative weight to the observations that are more likely to have attrition \citep[see][]{wooldridge2007inverse}.\\
\indent Provided the estimated model predicts attrition, it is easy to consider an IPW scheme in which factors are extracted using the method in Section \ref{section:method} --it is sufficient to consider the IPW scheme when calculating the correlation matrices.

\subsubsection{Allowing for Correlated Factors} \label{section:correlated}
\noindent By construction, the method in Section \ref{section:method} does not allow factors to be correlated. Sometimes, however, economic theory or intuition suggest that two or more sets of measurements should be considered. It is possible to apply the process in Section \ref{section:method} to two different sets of measurements independently. If the first factor of the two sets of measurements are correlated, this procedure preserves the correlation between the two first factors. When extracting the rest of the factors for each set of measurement systems, one can make the system orthogonal to the first factor of both systems as in step 3. Thus, the first factors of the two systems will be correlated while the rest of the factors will not be correlated within or across measurement systems.\footnote{I thank Tim Kautz for pointing this out. His example is the following. Assume the researcher has two measurements systems: one for height and one for weight. It makes sense to allow correlation for the ``primary'' measures of height and weight, which would be the first factors in this case, because it is natural for height and weight to be correlated. Then, it is possible to make the rest of the systems orthogonal because the researcher is only willing to capture extra variation or information from the measures.}