%Input preamble
\input{HelperFiles/Resources}

\begin{document}

\title{Factor Analysis}
\author{Jorge Luis Garc\'{i}a\thanks{Department of Economics, the University of Chicago (jorgelgarcia@uchicago.edu).} $^{,}$\thanks{I thank Sneha Elango, Tim Kautz, and Bradley Setzler for helpful comments.}}
\date{First Draft: May 29, 2014 \\ This Draft: \today}
\maketitle


\begin{abstract}
\noindent The objective of this document is to describe a step-by-step methodology to extract factors, or underlying latent variables, from a set of observed measures. Factor analysis is arbitrary by construction. Thus, I intend to provide the exact steps for factor analyzing a set of measures and elect a consistent way of being arbitrary. The theoretical fundamentals of this document come from \citet{gorsuch1983factor}. Although this author (or any other author) does not suggest any method over the other, I justify why the methods I elect are simple and transparent. Finally, I provide implementations in Python and Stata in separate files.
\end{abstract}

%Diagonal Factor Analysis
\section{Diagonal Factor Analysis} \label{section:dfa}
\input{Sections/factors/setting}
\input{Sections/factors/notation}
\input{Sections/factors/method}
\input{Sections/factors/number}
\input{Sections/factors/annotations}

%Rotating the Factors Analysis
\section{Why Rotation Makes sense and How to Go about it?} \label{section:rfa}
\noindent In Section \ref{section:correlated} I discuss how to allow for correlated factors. The procedure implies being certain about having two sets of measurement systems clearly devoted to factors of interest. For example, a measurement system could be devoted to ``weight'' and another to ``height''. The researcher arbitrarily chooses the measures with which she constructs the ``weight'' factor and the measures with which she constructs the ``height'' factor.\\
\indent It could be the case that the researcher does not want to take any stand on what the measurement systems are. This could happen for two reasons: (i) the researcher does not want to take arbitrary stands on what the dedicated measurement systems are; (ii) the researcher has no idea on what the dedicated measurement systems are.\\
\indent If this is the case, the alternative method is the following: (i) decide on a set of items composing the measurement system; (ii) set the number of factors (e.g.,  through a procedure like the one in  \ref{section:nfactors}); (iii) factor analyze the measures (e.g.,  through a procedure like the one in \ref{section:method}); (iv) rotate the factor axes to gain interpretability of the factors.\\
\indent The first three steps are clear: the researcher decides what the measurement system is, decides the number of factor she wants to use, and factor analyses the measures according to her preferred method. Rotation is convenient in this context. In general, after factor analyzing a set of measures, the output does not have an intuitive structure. Most items load on the first few factors that explain the greatest proportion of variance. Rotation is a linear transformation intending to give a ``simple structure'' to the factor system. In rough terms, it seeks a structure in which items load strongly on one factor and weakly in the rest of the factor.\footnote{\citet{abdi2003factor} explains the mathematical requirements for a structure to be simple and the intuition behind them.} This is the way in which the researcher may be able to associate each factor to an object of economic interest, ``height'', ``weight'', etc.\\ 



%\clearpage
\bibliographystyle{chicago}
\bibliography{BibtexFiles/Resources}
\clearpage

\end{document}